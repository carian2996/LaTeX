% Técnicas de muestreo
% Ian Castillo Rosales
% 26052014 - 29052014

\documentclass[a4paper]{article}
\title{tecnicas de muestreo}

\usepackage[spanish]{babel}
\usepackage[utf8]{inputenc}
\usepackage{hyperref}

\begin{document}

\begin{titlepage}
\newcommand{\HRule}{\rule{\linewidth}{0.5mm}}

\center 

\textsc{\LARGE Universidad Nacional Autónoma de México}\\[0.5cm]
\textsc{\Large Facultad de Estudios Superiores Acatlán}\\[0.5cm]
\textsc{\large Licenciatura de Actuaría}\\[1.5cm]

\HRule \\[0.4cm]
{ \huge \bfseries Técnicas de muestreo: La percepción en la oferta laboral}\\[0.4cm]
\HRule \\[1.5cm]

\begin{flushleft} \large
\emph{Autores:}\\
Jean Michel\textsc{Arreola Trapala}\\
Ian \textsc{Castillo Rosales}\\
Juan Pablo \textsc{Equihua Linares}\\
Gloria \textsc{González Vargas}\\
Moises \textsc{Márquez Sánchez}\\
Ricardo \textsc{Mungia González}\\[1.5cm]
\end{flushleft}

\begin{abstract}
Este artículo tiene como finalidad presentar el estudio realizado para conocer la perspectiva de la comunidad actuarial en la \href{http://www.acatlan.unam.mx/}{Facultad de Estudios Superiores Acatlán} con respecto a la oferta laboral.
Para la elaboración de este estudio se utilizaron técnicas de muestreo estudiadas en la asignatura de Muestreo, impartida por el profesor \href{http://www.sites.google.com/site/mahilhm/}{Mahil Herrera} con la valiosa colaboración del profesor invitado Moises.\\[1.5cm]
\end{abstract}

{\large \today}\\

\vfill
\end{titlepage}

\section{Introducción}
El documento presenta la realización de un estudio con base a los conocimientos adquiridos en la materia de Muestro durante el semestre 2014-2.\\

En la sección de \textit{El muestro y la profesión actuarial}, daremos una breve reseña del muestro como una rama de la estadística, cuya finalidad es la obtención de una muestra para su estudio y previo análisis, extender las concluciones a una población. El estudio se basa en la percepción de la oferta laboral de la profesión actuarial por parte de la comunidad estudiantil de la licenciatura de Actuaría en la Facultad de Estudios Superiores Acatlán.
Resulta razonable dar a conocer la historia de la profesión actuarial, principalmente en México, puesto que la percepción por parte de los estudiantes no ha sido la misma a lo largo de la historia de México. Presentamos algunas estadísticas de la oferta laboral existente en México para la profesión actuarial hoy en día para poder comparar los datos brutos contra la percepción de los estudiantes de la carrera de Actuaría.\\

En la segunda parte del documento, presentamos en la sección \textit{Una percepción laboral de los estudiantes de Actuaría en la FES Acatlán} la metodología usada para el estudio de la percepción de la oferta laboral en México por parte de los estudiantes de la licenciatura en Actuaría en la FES Acatlán. El propósito principal del estudio reside en comparar la situación actual de la profesión actuarial contra la apreciación que se tiene al estudiar esta licenciatura.
Presentaremos las técnicas correspondientes para la definición de la población a estudiar, la metodología para la obtención de la muestra y presentaremos los resultados obtenidos.\\

En la última sección daremos las concluciones de los resultados obtenidos.
 
\section{Objetivos y justificación}
\subsection{Objetivos generales y específicos}
Resulta de gran importancia establecer los objetivos que se tratarán de alcanzar en este estudio. En primer lugar, tenemos los \textbf{objetivos generales}:

\begin{enumerate}
	\item Aplicar los conocimientos adquiridos en la materia de Muestreo
	\item Justificar la visión que tienen los estudiantes sobre la oferta laboral en su área de estudio
\end{enumerate}

Con respecto a los \textbf{objetivos específicos} tenemos:

\begin{enumerate}
	\item Aplicar las ténicas más convenientes para la extracción de una muestra y poder generalizar los resultados en una población
	\item Comparar la oferta laboral existente en México para la profesión actuarial y la percepción que tienen los estudiantes de la licenciatura en Actuaría de la Facultad de Estudios Superiores Acatlán.
\end{enumerate}

\subsection{Justificación}
La oferta laboral es un tema recurrente entre los jóvenes universitarios, desde el momento de elegir una carrera hasta los últimos días en la misma. Un factor influyente a la hora de estudiar un grado universitario es precisamente, la oferta laboral.\\

A lo largo de los estudios uno puede cambiar la perspectiva de la carrera, las oportunidad que presenta en sus diversas especialidad y los campos laborales en los que uno puede irse enfocando, es bien conocido que la licenciatura de Actuaría tiene una amplia oferta laboral desde sus inicios, los actuarios pueden desempeñars en cualquier aspecto que presente riesgos, que se deseen mitigar, controlar o simplemente, medir.

Es muy importante estar concientes de la oferta laboral que existe, ya uno puede sentirse muy seguro con los estudios que se poseen, pero muchos factores influyen a la hora de hacer el gran salto de la universidad al mundo laboral.\\

Es por estas razones que nuestro estudio pretende identificar la percepción que tienen los jóvenes de la licenciatura de Actuaría sobre la oferta que actualmente está mientras ellos estudian, esto nos dará una clara comparación acerca de que es lo que esperan los jóvenes actuarios contra lo que se ofrece en el mercado.
 
\section{El muestro y la profesión actuarial}
Es importante para nuestro estudio conocer las diferentes técnicas que se requieren para poder extraer una muestra de la población a la cual vamos a analizar. Como se mencionó, uno de los objetivos generales de éste trabajo es el de aplicar los conocimientos adquiridos en técnicas de muestreo y concideramos necesario dar una breve introducción para que más adelante podamos aterrizar todos los conceptos que se necesitan para el estudio.\\

El muestreo tiene un objetivo muy claro, extraer (de la "mejor" forma posible) una muestra representativa de una población, de donde podamos generalizar los resultados obtenidos en ella. \cite{lohr}

Consideramos de igual manera, poner en contexto la situación que estaremos observando en la muestra, que es la oferta laboral de la profesión actuarial en México. Por lo tanto, daremos una pequeña reseña de la historia de la profesión actuaria, su incursión en México y la situación actual de la oferta laboral en nuestro país.

\subsection{El muestro. Una breve historia}
En estadística se conoce como muestreo a la técnica para la selección de una muestra a partir de una población. Este proceso permite ahorrar recursos, y a la vez obtener resultados parecidos a los que se alcanzarían si se realizase un estudio de toda la población.\\

Existen dos métodos para seleccionar muestras de poblaciones: el muestreo no aleatorio y el muestreo aleatorio (que incorpora el azar como recurso en el proceso de selección). Cuando éste último cumple con la condición de que todos los elementos de la población tienen alguna oportunidad de ser escogidos en la muestra, si la probabilidad correspondiente a cada sujeto de la población es conocida de antemano, recibe el nombre de muestreo probabilístico. 
Una muestra seleccionada por muestreo no aleatorio puede basarse en la experiencia de alguien con la población. Algunas veces una muestra no aleatoria se usa como guía o muestra tentativa para decidir cómo tomar una muestra aleatoria más adelante.\cite{wiki}\\

Según Rao (2005) \cite{rao}, el primer personaje interesado en el método de muestreo fue el estadístico noruego A. N. Kaier (1897) puesto que demostró empíricamente que seleccionanado muestras estratificadas se obtienen mejores resultados en los estimativos de medias y totales. En 1906, Bowley utiliza aproximaciones a la distribución normal para la estimación de proporciones y propone la fórmula de la estimación de la varianza para diseños de muestreo estratificados.\\

Para la década de 1920, el método representativo era usado de manera difundida en Estados Unidos y alrededor del mundo. Fue así coo en 1924, el Instituro Internacional de Estadística crea una comisión de discusión de este método. Los resultados del comite incluyen el trabajo de Bowley (1926) basado en métodos de selección representativos con probabilidades de inclusión iguales. COn estos avances teóricos y con la publicidad de tablas de números aleatorios por Tippett (1927) se facilitó la selección de muestras probabilísticas.

\subsection{La profesión actuarial}
La palabra actuario se deriva del latín \textit{actuarius}, la cual fue enmpleada durante la época del Imperio Romano y se refería a diversas profesiones. En lo civil, se designaba a las personas que levantaban actas del Senado o que intervenía en diferentes actos oficiales, tales como matrimonio, nacimientos, etcétera.\\

En el año 1774, la compañía \textit{The Equitable} usó por primera vez la palabra actuario cuando contrató al matemático W. Morgan. Las compañías inglesas designaban al actuario como la persona encargada de la contabilidad y su tarea principal era la de calcular las tarifas y reservas. La palabra actuario fue introducida por primera vez en la ley inglesa de 1819, cuando se prohibía a las sociedades mutuas el uso de tablas y estadísticas no aprobadas por dos o más personas designadas como actuarios. Los demás países europeos fueron adoptando la palabra de actuario en el mismo sentido que lo hacian los ingleses.

\subsection{Actuaría en México y su oferta laboral}

\section{Una percepción de la oferta laboral por los estudiantes de Actuaría en la FES Acatlán}

\subsection{Instrumento}
\subsection{La población}
\subsection{El marco muestral}
\subsection{La técnica}
\subsection{La muestra}

\section{Los resultados}

\section{Las concluciones}

\begin{thebibliography}{6}

\bibitem{coch}
  Cochran, W. G.
  \emph{Sampling techniques}.
  John Wiley and Sons, New York,
  3nd edition,
  1997.

\bibitem{lohr}
  Lohr, Sharon L.
  \emph{Sampling: Design and Analysis}.
  Cengage Learning,
  2nd edition,
  2009.
  
\bibitem{crespo}
  Sánchez Crespo, J. L.
  \emph{Curso intensivo de muestreo en poblaciones finitas}.
  Instituto Nacional de Estadística, Madrid,
  1995.
  
\bibitem{rao}
  Rao, J. N. K.
  \emph{Interplay between survey theory and practice: An appraisal}.
  Survey Methodology, 
  117-138,
  2005.
  
\bibitem{sudman}
  Sudman, Seymour
  \emph{Applied sampling}.
  Academic Press, USA,
  1997.
  
\bibitem{wiki}
  (n. d.)
  \emph{Muestreo (estadística)}.
  Wikipedia, 
  Recuperado el 26 de Mayo de 2014, 
  de http://es.wikipedia.org/wiki/Muestreo\_(estadística)

\end{thebibliography}

\end{document}