%----------------------------------------------------------------------------------------
%	Paquetes y otras configuraciones
%----------------------------------------------------------------------------------------

\documentclass[margin]{res}
\usepackage{helvet}
\usepackage[spanish]{babel}
\usepackage[utf8]{inputenc}
\setlength{\textwidth}{5.1in}
\title{cv}

\begin{document}

%----------------------------------------------------------------------------------------
%	Nombre y dirección
%----------------------------------------------------------------------------------------

\moveleft.5\hoffset\centerline{\large\bf Ian Castillo Rosales}
 
\moveleft\hoffset\vbox{\hrule width\resumewidth height 1pt}\smallskip
 
\moveleft.5\hoffset\centerline{Dr. Nicolás León 173} % Your address
\moveleft.5\hoffset\centerline{Jardín Balbuena, Distrito Federal 15900}
\moveleft.5\hoffset\centerline{(5255) 5768-5588 o (04455) 2132-5415}

%----------------------------------------------------------------------------------------

\begin{resume}

%----------------------------------------------------------------------------------------
%	Objetivo
%----------------------------------------------------------------------------------------
 
\section{OBJETIVO}  

Lograr un desarrollo sólido en el área de derivados financieros a través de mis conocimientos y habilidades. Llegar a ser un factor de eficiencia, desarrollo y crecimiento dentro de la institución. Adquirir la formación necesaria para alcanzar siempre las metas que se proponen, de una manera concisa y exitosa.

%----------------------------------------------------------------------------------------
%	Experiencia profesional
%---------------------------------------------------------------------------------------- 
\section{EXPERIENCIA \\ PROFESIONAL}

{\sl Becario del Programa de Servicio Social}\\
Banco de México, Febrero 2014 - Agosto 2014\\
Subgerencia de Información de Moneda Extranjera y Derivados
\begin{itemize} \itemsep -2pt
\item Apoyo en la elaboración de recursos necesarios para el modelo de validación de información en operaciones con derivados y con moneda extranjera del sistema financiero mexicano.
\item  Programación de procesos en Lenguaje R y visual Basic Application (VBA) Excel.
\end{itemize}

%----------------------------------------------------------------------------------------
%	Educación
%----------------------------------------------------------------------------------------

\section{EDUCACIÓN}

{\sl Especialidad en Data Science, Curso Online} \hfill Presente\\
John Hopkins University\\
Bloomberg School Of Public Health\\
Certificación Oficial

{\sl Licenciatura en Actuaría} \hfill 2010 - 2014 \\
Pasante\\
Universidad Nacional Autónoma de México\\
Facultad de Estudios Superiores Acatlán

%----------------------------------------------------------------------------------------
%	Experiencia Académica
%---------------------------------------------------------------------------------------- 

% \section{EXPERIENCIA \\ ACADÉMICA}

% {\sl Ponente} \hfill Agosto 2013\\
% Encuentro Estudiantil de Usuarios de R\\
%  Instituto Tecnológico Autónomo de México (ITAM)

% {\sl Adjunto de profesor} \hfill Agosto 2012 - Mayo 2013\\
% Calculo Diferencial e Integral I, Álgebra Superior II\\
% Facultad de Estudios Superiores Acatlán\\

%----------------------------------------------------------------------------------------
%	HABILIDADES COMPUTACIONALES
%----------------------------------------------------------------------------------------

\section{LENGUAJES \\ SOFTWARE} 
\begin{tabular}{ l l }
  Lenguaje R & \hspace{1cm} Avanzado\\
  VBA for Excel & \hspace{1cm} Avanzado\\
  Lenguaje SAS & \hspace{1cm} Intermedio\\
  MySQL & \hspace{1cm} Intermedio\\
  C/C++ & \hspace{1cm} Intermedio\\
  Python & \hspace{1cm} Básico\\[0.2cm]
  Office & \hspace{1cm} Avanzado\\
  Software econométrico & \hspace{1cm} Avanzado\\
  Software estadístico & \hspace{1cm} Avanzado\\
\end{tabular}

%----------------------------------------------------------------------------------------
%	Idiomas
%---------------------------------------------------------------------------------------- 

\section{LENGUAJES}
\begin{tabular}{ l l }
  Inglés & \hspace{2.9cm} Avanzado\\
  Portugués & \hspace{2.9cm} Intermedio\\
  Finés & \hspace{2.9cm} Básico\\
\end{tabular}

%----------------------------------------------------------------------------------------
%	Actividades extracurriculares
%----------------------------------------------------------------------------------------

\section{OTRAS \\ ACTIVIDADES} 
Asistente en la Primera Escuela Latinoamericana de Estadística Bayesiana organizada por la International Society for Bayesian Analysis (ISBA) en la Escuela de Estadística de la Universidad de Costa Rica. Julio 2013.\\[0.2cm]
Asistente en el Primer Simposio Centroamericano de Estadística Bayesiana organizado por la Escuela de Estadística de la Universidad de Costa Rica. Julio 2013.

%----------------------------------------------------------------------------------------

\end{resume}
\end{document}